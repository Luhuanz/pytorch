\section{实验结果分析}
 \subsection{总结}
 Mnist手写数据集本身不是一个复杂的数据集,而且它是单通道的图片,因而建模起来并不复杂。如果数据增强做得好点的话,train部分是可以训练到100\%。但是过多的epoch也可能会导致模型过拟合,从而在test上面acc下降,例如ResNet18和ResNet34,ResNet18的4个epoch平均的acc为99.1\%,而ResNet34却只有98\%。可能就是train过拟合了。在5个网络里,其中训练效果最好的是AlexNet,它的模型效果达到99.26\%。可能也是因为并不复杂的原因。考察5个网络,发现均能超过97\%,可见效果还不错。
 \subsection{改进模型}
 可以进一步做数据增强,fastai中提供了mixup混合形式,这样使得图片直接更加独立。对于googLeNet可以将5x5的卷积核换成3x3的,并且使用1x1降维。不要Resize它的尺寸,直接建模。因为Resize可能会损失图像的一些特征。
 
  